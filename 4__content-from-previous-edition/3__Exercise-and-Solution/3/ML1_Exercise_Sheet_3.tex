\documentclass[]{scrartcl}

\usepackage{amssymb}
\usepackage{amsmath}
\usepackage{hyperref}
\usepackage{enumitem}
\usepackage{mathtools}
\usepackage[ddmmyyyy]{datetime}
\renewcommand{\dateseparator}{.}

\newcommand{\R}{\mathbb{R}}
\newcommand{\N}{\mathbb{N}}
\newcommand{\bx}{\mathbf{x}}
\newcommand{\ba}{\mathbf{a}}
\newcommand{\bb}{\mathbf{b}}
\newcommand{\bv}{\mathbf{v}}
\newcommand{\bw}{\mathbf{w}}
\newtheorem{prop}{Proposition}
\newtheorem{lem}{Lemma}
\setlist[enumerate,1]{label=\bfseries\arabic*)}

\author{Prof. Marius Kloft \and TA: Billy Joe Franks}
\title{Machine Learning I: Foundations \\ Exercise Sheet 3}
\date{\today\\Deadline: 19.05.2020}
\begin{document}
\maketitle

\begin{enumerate}

\item \textbf{(MANDATORY) 10 Points}\\ 
Suppose that $k_1,\ldots,k_n: \R^d \times \R^d \to \R$ are kernels. Let $c_1, \dots, c_n \in \R^+$ and $p \in \N$. Prove that the following functions $k$ are also kernels.
\begin{enumerate}
	\item \textbf{Scaling}: $k(\bx,\bx') := c_1 k_1(\bx,\bx')$
	\item\textbf{Sum}: $k(\bx,\bx') := k_1(\bx,\bx') + k_2(\bx,\bx')$
	\item\textbf{Linear combination}: $k(\bx,\bx') := \sum_{i=1}^n c_i k_i(\bx,\bx')$
	\item\textbf{Product}: $k(\bx,\bx') := k_1(\bx,\bx')k_2(\bx,\bx')$
	\item\textbf{Power}: $k(\bx,\bx') := k_1(\bx,\bx')^{p}$
\end{enumerate}

\item In the lecture a few kernels were proposed, and here we will prove them to be kernels. Prove the following statements:
\begin{enumerate}
	\item \textbf{Polynomial kernel}: $k(\bx, \bx'):=(\bx^T\bx'+c)^d$ is a kernel. 
  \item\textbf{Limits}: If $k_i: \R^d \times \R^d \to \R$, $i\in\N$, are kernels and $k(\bx,\bx'):=\lim_{n \to \infty}k_n(\bx, \bx')$ exists for all $\bx, \bx'$, then $k(\bx, \bx')$ is a kernel. Use the definition of positive semi-definiteness. 
	\item\textbf{Exponents}: If $\tilde{k}$ is a kernel, then $k(\bx,\bx') := \exp{(\tilde{k}(\bx,\bx'))}$ is a kernel.
	\item\textbf{Functions}: If $\tilde{k}$ is a kernel and $f:\R^d \to \R$ then $k(\bx,\bx') := f(\bx)\tilde{k}(\bx,\bx')f(\bx')$ is a kernel.
	\item\textbf{Gaussian RBF kernel}: $ k(\bx, \bx'):=\exp{(-\frac{\left\lVert\bx-\bx'\right\rVert}{2})}$ is a kernel.
\end{enumerate}
\textbf{Hint:} Use the results from Exercise 1 above.

\newpage
\item Prove the following lemma:
	\begin{lem}
		Let $V$ be a vector space and $I$ a set. Let $f_i:V \to \R$ be a collection of functions indexed by $i \in I$. If $f_i$ is convex for all $i$, then the function
		$$f(x) = \max_{i \in I} f_i(x)$$
		is also convex.
	\end{lem}
	
\item Solve programming task 3.
\end{enumerate}
\end{document}